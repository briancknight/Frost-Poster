%%%%%%%%%%%%%%%%%%%%%%%%%%%%%%%%%%%%%%
% LaTeX poster template
% Created by Nathaniel Johnston
% August 2009
% http://www.nathanieljohnston.com/2009/08/latex-poster-template/
%%%%%%%%%%%%%%%%%%%%%%%%%%%%%%%%%%%%%%

\documentclass[final]{beamer}
\usepackage[scale=.9]{beamerposter}
%\usepackage[scale=1.24]{beamerposter}
\usepackage{graphicx}	% allows us to import images
\usepackage{subcaption}
\usepackage[export]{adjustbox}
\usepackage{booktabs}
\usepackage{multirow}
\usepackage{soul}

%-----------------------------------------------------------
% Define the column width and poster size
% To set effective sepwid, onecolwid and twocolwid values, first choose how many columns you want and how much separation you want between columns
% The separation I chose is 0.024 and I want 4 columns
% Then set onecolwid to be (1-(4+1)*0.024)/4 = 0.22
% Set twocolwid to be 2*onecolwid + sepwid = 0.464
%-----------------------------------------------------------

\newlength{\sepwid}
\newlength{\onecolwid}
\newlength{\twocolwid}
\newlength{\threecolwid}
\setlength{\paperwidth}{48in}
\setlength{\paperheight}{36in}
\setlength{\sepwid}{0.024\paperwidth}
\setlength{\onecolwid}{0.22\paperwidth}
\setlength{\twocolwid}{0.464\paperwidth}
\setlength{\threecolwid}{0.708\paperwidth}
\setlength{\topmargin}{-0.5in}
\usetheme{confposter}
\usepackage{exscale}
\usepackage[font=scriptsize,labelfont={color=Blue,bf}]{caption}

%-----------------------------------------------------------
% The next part fixes a problem with figure numbering. Thanks Nishan!
% When including a figure in your poster, be sure that the commands are typed in the following order:
% \begin{figure}
% \includegraphics[...]{...}
% \caption{...}
% \end{figure}
% That is, put the \caption after the \includegraphics
%-----------------------------------------------------------

\usecaptiontemplate{
\small
\structure{\insertcaptionname~\insertcaptionnumber:}
\insertcaption}

%-----------------------------------------------------------
% Define colours (see beamerthemeconfposter.sty to change these colour definitions)
%-----------------------------------------------------------

\setbeamercolor{block title}{fg=ngreen,bg=white}
\setbeamercolor{block body}{fg=black,bg=white}
\setbeamercolor{block alerted title}{fg=white,bg=dblue!70}
\setbeamercolor{block alerted body}{fg=black,bg=dblue!10}

%-----------------------------------------------------------
% Name and authors of poster/paper/research
%-----------------------------------------------------------

\title{Glacial Cycles and the 100 Kyr Problem}
\author{Brian Knight*, Raymart Ballesteros*; Advisor: Dr. Charles D. Camp}
\institute{Frost Summer Undergraduate Research 2018}

%-----------------------------------------------------------
% Start the poster itself
%-----------------------------------------------------------
%Figure count on / off
%\setbeamertemplate{caption}[numbered]
\begin{document}
\begin{frame}[t]
  \begin{columns}[t]												% the [t] option aligns the column's content at the top
    \begin{column}{\sepwid}\end{column}			% empty spacer column
    \begin{column}{\twocolwid}
    \begin{columns}[t,totalwidth=\twocolwid]
    \begin{column}{\onecolwid}
    	  \begin{alertblock}{Abstract}
    	  The Earth's climate during the Late Pleistocene era (1.25 Mya to 12.5 kya) is characterized by large glacial cycles: oscillations in the size of large land-based ice sheets on a 100 kyr time scale.The underlying cause for these long cycles and the predictability of their timing remain open questions. Numerous models have been proposed to explain the observed behavior including those based on forced dynamical systems in which the observed cycles are the result of complicated interactions between the quasi-periodic astronomical forcing and internal free oscillations. The overall timing of the cycles is primarily controlled by nonlinear synchronization - a ubiquitous feature of forced dynamical systems. The strong asymmetry exhibited between the warming and cooling phases of a cycle can be associated with an asymmetry in the predictability of behavior within a glacial cycle. In the work, we explore these behaviors of forced dynamical systems using two models of the glacial cycles. 
    	  	%There is an ongoing issue in the study of Paleoclimate regarding the climactic causes of the glacial cycles, or oscillations of ice growth and ice decay. In our data there is behavior that has yet to be understood. What is known is that global climate change is "paced" by astronomical forcing, or the amount of insolation the earth receives from the sun. With this understanding people have developed many models to try to reproduce the empirical data. The complexity of the system introduces the concept of Non-Linear Synchronization, causing an ubiquitous predictability in the timings of glacial  We first discuss this setback, and then introduce two conceptual models along with appropriate tools to compare and contrast between model and data. Specifically, we explore the idea of asymmetry between the phases of a system, and the predictability of a given cycle, or oscillation, in the presence of external forcing. 
    	  \end{alertblock}
    	  \vskip1ex
    	  \begin{block}{Introduction/Background}
    	  	%There are several physical ideas that motivate research in this area, derived from empirical data measured from ice and sediment cores. There is a clear cyclical nature in global ice volume and global temperature. Some parts that are not well understood:
    	  	%descripion of data,
    	  	Variations in $\delta$\textsuperscript{18}O records -- an isotopic based proxy measurement for global ice volume -- shown below:	
    	  	%plot,
    	  	%list of key features of importance,
    	  	%parts we don't understand
    	  	%language: we see glacial cycles; oscillations in the growth and decay of large land based ice sheets
    	  	%transition to discussion of behavior of models
      	\begin{center}
      	\begin{figure}
			\includegraphics[width=10in]{DO18.png}
		\caption{$\delta$\textsuperscript{18}O records for past 2.5 million years, red arrow points to higher temp, yellow points to present.}
		
		%\label{1}
		\end{figure}
		\end{center}
		In this $\delta$\textsuperscript{18}O record we observe glacial cycles, or oscillations in the growth and decay of large land based ice sheets, of a varying timescale.
		What we understand: 
			\begin{itemize}
			\small \item Early 40 kyr timescale oscillations due to astronomical forcing (obliquity)
			\small \item General cooling trend	
			\end{itemize}
		What we do not understand: 
			\begin{itemize}
    	  		\small \item Transition from fast cycles (40 kyr timescale, to slow, 100 kyr timescale around 800 kyr ago)
    	  		\small \item Why realized timescale of 100 kyr?
    	  		\small \item Simply a bi-stable system? Other possible dynamics?
    	  	\end{itemize}
		To further study these questions we make use of two conceptual models. The approach is to:
		\begin{itemize}
		\small \item Collect model output which simulates Late Pleistocene cycles
		\small \item Investigate the interaction between internal and external dynamics
		\small \item Distinguish predictability of timing versus predictability of state within a cycle.
		\end{itemize}
		%$\delta$\textsuperscript{18}O is a proxy measurement for global ice volume. Around 800 kyr ago the cycles begin to oscillate on a 100 kyr timescale. These are coined "Late Cycles", and Our models are set up to replicate these late cycles only.
      \end{block}
     \end{column}
    \begin{column}{\sepwid}\end{column}       
    \begin{column}{\onecolwid}
    	  \begin{alertblock}{Core (Mathematical) Questions}
    	   \begin{itemize}
    	   	\item How does the interaction between an external quasi-periodic forcing signal and a internal limit cycle create the realized timescale of oscillation?
    	   	\item To what degree is the timing of the glacial cycles predictable?
    	   	\item What is the role of asymmetry in the internal dynamics?
    	   	\item How does the interaction between external and internal dynamics effect predictability within a glacial cycle?
    	   \end{itemize}
    	  \end{alertblock}
    	  \vskip1ex
      \begin{block}{Non-Linear Synchronization}
      	%Non-linear synchronization, in the context of the dynamics of our models, implies the timing of cycles is not an robust way to validate the model. In the plot below we see how all large warming events are paced by high obliquity events in the forcing signal.
      	Deglaciation events are paced by maximum obliquity (astronomical forcing). Some maximum obliquity events are skipped. Exploring how the conceptual models respond to the quasi-periodic forcing signal, we implement a circle plot to compare timing of events, namely ice de-glaciation and maximum obliquity.
      	
      	% Discuss how this figure demonstrates how our model might synchronize to our forcing signal. Talk about skipped obliquity or every 3rd obliquity maximum pacing these cycles. Synchronize to a subset of all possible timings. Discuss ubiquity of sychronization to quasi-periodic forcing. SM90 has a similar result.
      	
      	%
       \begin{figure}
       \centering
      	\includegraphics[width=10in]{LaskarObl_PP04_cirlceplot_signalonly.jpg}
      	\caption{\textbf{top}: Global Ice Volume Output from PP04 Model Run-red dot placed in middle of full melt, \textbf{middle}: Laskar Obliquity Signal, red dots plotted at same time step, \textbf{bottom}: (left to right) PP04, SM90, Data cirleplots to determine whether the two signals are in phase. Red dots on the mid-line of the plot are perfectly in phase, those to the left are leading.}
      	%\label{3}
       \end{figure}
       \begin{figure}
       \centering
      	\begin{subfigure}{.33\textwidth}
      	\centering
      	\includegraphics[scale=.17]{LaskarObl_PP04_cirlceplot_only.jpg}
      	\caption{PP04}
      	\end{subfigure}%
      	\begin{subfigure}{.33\textwidth}
      	\centering
      	\includegraphics[scale=.32]{sm90_insolLaskar_oblpacing_circleplot_only.jpg}
      	\caption{SM90}
      	\end{subfigure}%
      	\begin{subfigure}{.33\textwidth}
      	\centering
      	\includegraphics[scale=1]{Inkedtemp800_circle_only.jpg}
      	\caption{Data}
      	\end{subfigure}%
       \end{figure}
      	Note: figs (b) and (c) are analogous plots for SM90 and data; each is significantly paced by the obliquity signal, typically every 3rd max.
%       To note: 
%       \begin{itemize}
%       \small \item Figures (b) and (c) show the analogous results for SM90 and the data respectively. 
%       \small \item Each is significantly paced by the obliquity signal. 
%       \end{itemize}
       \begin{figure}
        		\includegraphics[width=10in]{Synchronization_Ice_cropped.jpg}
        		\caption{Synchronization* of 1000 random IC for PP04 in Global Ice Volume. \textbf{*note}: left is present, right is past, time runs right to left; cool temp is low, high temp is high. Skipped obliquity}
       \end{figure}
       Trajectories from different initial conditions generally synchronize to three stable clusters, identified by which obliquity maximum triggers a full melt. Short-lived desychronization events occur spoaridcally.
       %The synchronization sometimes goes out of phase when a trajectory triggers a full melt out of phase with the rest of the cluster.
       %very analogous to \textit{De Saedeleer's Paper}. 
       
       % Broad discusson: Skipped obliquity, 3 clusters of trajectories with periods of desynchronization, possible thate one cluster becomes very unlikely/less dence due to the quasi-periodic forcing. Tziperman and De Saedeleer holds for our model.
     
      %\textit{The model synchronizes to three separate stable trajectories, very analogous to work done by De Saedeleer with simpler systems. The three distinct trajectories correspond to the three different behaviors in the slow phase, namely: no defective melts, one defective melt, two defective melts.}
      \end{block}
    \end{column}
   \end{columns}
   \begin{block}{Conceptual Models}
   \end{block}
   \begin{columns}[t,totalwidth=\twocolwid]
   \begin{column}{\onecolwid}
       Saltzman \& Maasch 1990 (SM90)[\textit{Smooth}]       
       \begin{itemize}
        \small{
        \item $ dX/dt = -X - Y - vZ - uR(t*) + W_X(t*)$, \mbox{$dY/dt = -pZ + rY + sZ^2 - wYZ - Z^2Y + W_Y(t*)$}, \mbox{$dZ/dt = -q(X+Z)+W_Z(t*)$}
        \item Smooth 3 Variable System: X = Global Ice Mass, Z = Deep Ocean Temp., \mbox{Y = Atmos. CO2}, non-dimensionalized.
        \item $W_X, W_Y, W_Z$ stochastic variables.
        \item Internal Stable Limit Cycle, Multi-Stable* with 3 stable trajectories.
        }
       \end{itemize}
       \vskip1ex
    \end{column}
    %\begin{column}{\sepcolwid}\end{column}
    \begin{column}{\onecolwid}
        Paillard \& Parrenin 2004 (PP04)[\textit{Non-Smooth}]
        \begin{itemize}
		\small{
		\item $ dV/dt = (V_R - V)/\tau_V $, $ dA/dt = (V - A)/\tau_A $, $ dC/dt = (C_R - C)/\tau_C $
        \item $ V_R = xC - y I_{60} + z $
        \item $ C_R = \alpha I_{65} - \beta V + \gamma H(-F) + \delta $
        \item $ F = aV - bA - cI_{60} + d $, d = 0.27 for late pleistocene simulation.
        \item $I_{65}, I_{60}$, solar insolation at 65, 60 deg. north latitude. 
        \item Also 3D system. \textbf{Non-smooth}. V = Global Ice Mass, A = Antarctic Ice Volume, \mbox{C = Atmos. C02},
        \item Internal Stable Limit Cycle, Multi-Stable* with 3 stable trajectories, Model Switch non-linearity (Heaviside function: H = 1 if F <0, H = 0 otherwise, F < 0: Relaxation Warming Model, F > 0: Relaxation Cooling Model)
        }
       \end{itemize}
      \vskip1ex
      \end{column}
     \end{columns} 
     \begin{center}
      %\vskip1ex
      \footnotesize *As observed in parameter space / forcing tested
     \end{center}
      \vskip2ex
      \end{column}
    \begin{column}{\sepwid}\end{column}
    \begin{column}{\onecolwid}
     \begin{block}{Predictability of Models}
     $\delta$\textsuperscript{18}O records of Late Pleistocene show a memory of state in ice decay that is not found in ice growth. In particular, there is a significant anti-correlation between the ice volume at a glacial maximum and at the subsequent glacial minimum. Knowing this and with understanding that model trajectories synchronize, we begin to search for predictability within a glacial cycle.
     %different characteristics in our model outputs. Namely, an anti-correlation in the ice variable between glacial maximums and proceeding minimums. 
      \begin{figure}
        \centering
        \label{something}
      	\begin{subfigure}{.5\textwidth}
      	 \centering
         \includegraphics[width=5in,height=2in]{model_out_run8_cropped.jpg}
         \caption{PP04 Model Output. \textbf{black}: Atmos. CO2, \textbf{blue}: Global Ice Volume, \textbf{red}: Antarct. Ice Sheet Volume.}
        \end{subfigure}%
      	\begin{subfigure}{.5\textwidth}
      	 \centering
         \includegraphics[width=5in,height=2in]{sm90_model_output_run11_cropped2.jpg}
         \caption{SM90 Model Output. \textbf{black}: Atmos. CO2, \textbf{blue}: Global Ice Mass, \textbf{red}: Deep Ocean Temp..}
        \end{subfigure}
        \begin{subfigure}{\textwidth}
        \begin{center}
        \includegraphics[width=8in,height=3in]{anti_corr_timeseries_cropped.jpg}
        \caption{\textbf{blue}: PP04: Global Ice Volume output, time going from right to left. \textbf{red dots}: maxima (low), and minima (high)of full melts. \textbf{black}: Two cubic splines through the maxima and minima of full melts to show the anti-correlation we suspected.}
        \end{center}
        \end{subfigure}%   
      \end{figure}
		In (c) above we observe the anti-correlation between glacial minima and maxima in our model output. To gain better, we observed projections of phase spaces of both models and collect specific measurements. For PP04:
		\begin{itemize}
		\small \item Trigger point implemented by a Heaviside function, marked by dashed line
		\small \item Measure ice volume at trigger points of model
		\small \item Measure max ice volume and min ice volume of cycle
		\end{itemize}
		%Defining In PP04 there is a built in non-linearity, a model switch implemented by a Heaviside function. This is convenient because it allows us to mark distinctly when we are in what phase. The dashed line in the 4 plots below represent a projection of the surface in the phase space on which the model switch occurs. We use these switching times to measure state variables, as well as the largest and smallest ice value for each cycle.
      	\begin{figure}
      	\centering
      	\begin{subfigure}{.5\textwidth}
      		\centering
      		\includegraphics[width=3.5in]{PP04_UFWarmTraj_cropped.jpg}
      		\caption{Unforced Warming Phases}
      		\label{4a}
      	\end{subfigure}%
      	\begin{subfigure}{.5\textwidth}
      		\centering
			\includegraphics[width=3.5in]{PP04_UFCoolTraj_cropped.jpg}
      		\caption{Unforced Cooling Phases}
      		\label{4b}
      	\end{subfigure}
      	\begin{subfigure}{.5\textwidth}
      		\centering
      		\includegraphics[width=3.5in]{PP04_FWarmTraj_cropped.jpg}
      		\caption{Forced Warming Phases, same IC as above}
      		\label{4c}
      	\end{subfigure}%
      	\begin{subfigure}{.5\textwidth}
      		\centering
      		\includegraphics[width=3.5in]{PP04_FCoolTraj_cropped.jpg}
      		\caption{Forced Cooling Phases, same IC as above}
      		\label{4d}
      	\end{subfigure}%     	
      	\end{figure}
      	Trajectories all moving counter-clockwise, a clear distinction can be made by observed plots a-d above. Unforced vs. Forced Behavior in warming phases is very similar in this projection, namely this parallel curve structure. Both the synchronization of the forcing and the asymmetry of the phases(average ratio of time spent cooling to time spent warming per cycle is approximately 6/1) contribute to this. In contrast, the cooling phase structure changes drastically from the unforced to the forced. Any parallel structure is that existed is erased by the quasi-periodic forcing .
      \vskip1ex
      For the smooth model, SM90, there is no definitive point at which the phase shifts because it is a continuous model. Instead, we look at the CO2 variable (Mu) and its derivative to mark what phase we are in.
      \begin{itemize}
      \small \item Trigger point chosen to be $\delta$CO2 = 0 
      \small \item High C02 implies lower ice mass, vice versa
      \small \item CO2 increasing: warming, decreasing: cooling
      \small \item Record state variables at trigger points as well as maximum and minimum ice masses
      \end{itemize}
      The choice of trigger point here is somewhat arbitrary. To ameliorate this we put emphasis on the maximum and minimum ice masses for comparisons.
        %That is, a higher CO2 measure is related to a lower ice mass; so as CO2 increases, ice mass decreases, thus marking a cooling phase (and similarly for warming phases).  These times dependent on the derivative on the CO2 variable were similarly used to measure state variables and the largest/smallest ice values for each cycle. This is a somewhat arbitrary choice, so we compare the results to the data analysis approach of measuring the global maximum and minimum ice measure for an a priori defined cycle.
      
    \end{block}
    
    \end{column}
    \begin{column}{\sepwid}\end{column}
    \begin{column}{\onecolwid}
    \begin{block}{Predictability Continued}
    \begin{figure}
      	\centering
      	\begin{subfigure}{.5\textwidth}
      		\centering
      		\includegraphics[width=3.5in]{sm90_unfor_ThetaI_ICplot_warming.jpg}
      		\caption{Unforced Warming Phases}
      		\label{4a}
      	\end{subfigure}%
      	\begin{subfigure}{.5\textwidth}
      		\centering
			\includegraphics[width=3.5in]{sm90_unfor_ThetaI_ICplot_cooling.jpg}
      		\caption{Unforced Cooling Phases}
      		\label{4b}
      	\end{subfigure}
      	\begin{subfigure}{.5\textwidth}
      		\centering
      		\includegraphics[width=3.5in]{sm90_insolLaskar_ThetaI_ICplot_warming.jpg}
      		\caption{Forced Warming Phases, same IC as above}
      		\label{4c}
      	\end{subfigure}%
      	\begin{subfigure}{.5\textwidth}
      		\centering
      		\includegraphics[width=3.5in]{sm90_insolLaskar_ThetaI_ICplot_cooling.jpg}
      		\caption{Forced Cooling Phases, same IC as above}
      		\label{4d}
      	\end{subfigure}%     	
      	\end{figure}
      	%In the unforced versions of these cycles we anticipate a significant anti-correlation in the warming phases, i.e. this parallel structure. Robust correlations are a key next step for future work. We also see a qualitative anti-correlation in the unforced cooling phases. However, this behavior in the slow part of the cycles is more easily erased by external forcing, as seen in the forced trajectories. In PP04, the systems memory of the Ice Volume when a cooling was triggered is lost as the forcing has time and amplitude to displace the unforced trajectory that otherwise will converge to the limit cycle. Hence the inherent asymmetry of these systems corrupts the predictability within cooling phases, but fails to do so in the warming phases. We see a more complicated behavior in SM90 that we leave to be explored further.
      	%allows for this anti-correlation in the warming to persist and corrupts the anti-correlation found in the cooling.
      	Qualitatively, these phase planes are similar to that of PP04.
      	\begin{itemize}
      	\small \item More time is spent in the cooling phase than the warming (appr. 3/1), thus cooling trajectories converge to stable limit cycle
      	\small \item Unforced cooling also similar, though less displacement
      	\small \item Little qualitative change in unforced warmings and forced warmings
      	\end{itemize}
     \end{block}
     \begin{block}{Conclusions \& Discussion}
     Taking specific forced model runs from both of these models, we find the correlations discussed in the phase plane figures. Below is a table containing correlations between phases of cycles. We also compare these correlations to the analogous sediment core data(composed of roughly 10 glacial cycles).
     For each model we find correlations for two warming/cooling phase definitions: \textbf{1.} Ice Variable Extrema, i.e. Ice max to Ice min as a warming phase, \textbf{2.} Model Trigger, as defined in the phase plots.
     \vskip1ex
	\small{\begin{table}[ht]
		\begin{tabular}{@{\extracolsep{4pt}}llccccccc} 
 		 {Warming/Cooling, Phase Definition} & {} & PP04(\textbf{\textit{f}}) & SM90(\textbf{\textit{f}}) & Data\\ 
   		 \hline
      	 \multirow{2}{*}{Warming: Ice Var. Extrema}
 		 & corr &  \textbf{-0.48} & -0.64 & -0.62\\ 
  		 & p-val & \textbf{<0.01} & <0.0001 & 0.10\\ 
  		 \hline
  		 \multirow{2}{*}{Cooling: Ice Var. Extrema}
  		 & corr & -0.07 & -0.34 & 0.39\\ 
  		 & p-val & 0.71 & 0.02 & 0.39\\ 
  		 \hline
  		 \multirow{2}{*}{Warming: Trigger}
  		 & corr & \textbf{-0.60} & -0.64 & N/A\\ 
  		 & p-val & \textbf{<0.001} & <0.00001 &  N/A\\ 
  		 \hline
  		 \multirow{2}{*}{Cooling: Trigger}
  		 & corr &  -0.06 & -0.49 & N/A\\ 
  		 & p-val & 0.75 & <0.001 & N/A\\ 
 		 \hline
		\end{tabular}
		\caption{Variable Correlations and Significance. \textbf{\textit{(f)}} denotes stats for forced model run.}
	 \end{table}}
	 \normalsize
	 %Where Ice max and Ice min refer to the highest and lowest amounts of ice in a cycle, and Ice-melt and Ice-cool denote the ice quantity at the respective trigger points of the model.
	 \vskip1ex
	 In both the non-smooth, PP04, and the smooth, SM90 conceptual models, the asymmetric limit cycles in the presence of a quasi-periodic forcing create strong anti-correlations in the warming phases. In PP04 any correlation in the unforced cooling phases is erased by this forcing, whereas in SM90, a complex geometry creates strong anti-correlation as it cools, something to be further explored. \textit{The role of asymmetry in a system is a key part of the realized response to external forcing}.
	 \vskip1ex
	 \textit{The pacing of the cycles by the obliquity signal plays a large role in the realized 100 kyr oscillation in the two conceptual models, as well as synchronizing the forcing signal itself}. Overall, the interaction between internal and external dynamics tends to diminish the predictability of certain phases while having little effect on others.
	 %and no significant correlations in the cooling phases. This characteristic should be considered when building higher dimensional models.
    \end{block}
    %\vskip1ex
    \begin{block}{Acknowledgments}
    *Frost Research Fellows, recipients of Frost Undergraduate Student Research Awards, funded by the Bill and Linda Frost Fund.
    \end{block}
    %\vskip1ex
    \begin{block}{References}
		        \tiny{\begin{thebibliography}{99}
		        \bibitem{PP04} D. Paillard, F. Parrenin, The Antarctic ice sheet and the triggering of deglaciations, Earth and Planetary Science Letters \textbf{227} (2004) 263-271.
		        \bibitem{SM90} B. Saltzman, K. A. Maasch, A first order global model of late Cenozoic climactic change, Transactions of the Royal Society of Edinburgh; Earth Sciences, \textbf{81}, (1990), 315-325.
		        \bibitem{tzip06} Tziperman, E., M. E. Raymo, P. Huybers, and C. Wunsch (2006), Consequences of pacing the Pleistocene 100 kyr ice ages
by nonlinear phase locking to Milankovitch forcing, Paleoceanography, 21, PA4206, doi:10.1029/2005PA001241.
		        \bibitem{desaed} B. De
Saedeleer, M. Crucifix, S. Wieczorek, Is the astronomical forcing a reliable and unique pacemaker for climate? A
conceptual model study, Climate Dynamics, \textbf{40}, (2013), 273-294.
		        %\bibitem{KLPL06} D.~W. Kribs, R. Laflamme, D. Poulin, M. Lesosky, Quantum Inf. \& Comp. \textbf{6} (2006), 383-399.
		        %\bibitem{zanardi97} P. Zanardi, M. Rasetti, Phys. Rev. Lett. \textbf{79},  3306 (1997).
		        \end{thebibliography}}
			      \vspace{0.75in}
    \end{block}
    \end{column}
 \end{columns}
\end{frame}
\end{document}
